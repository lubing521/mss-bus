
\documentclass[a4paper,12pt]{article}
\usepackage{latexsym}
\usepackage{amsfonts}
\usepackage{relsize}
\usepackage{indentfirst}
\usepackage[MeX]{polski}
\usepackage[latin2]{inputenc}
\usepackage[polish]{babel}

\title{
    \textbf{Sieci komputerowe 2 \\ Laboratorium} \\
    {\normalsize<<Ćwiczenie 1/5: Implementacja protokołu warstwy 2 na wspólnej szynie>>} \\
    {\large{>>Dokumentacja końcowa<<}}
}

\author{
    \makebox[8em][c]{Piotr Bałut} \and
    \makebox[8em][c]{Maciej Rubikowski} \and
    \makebox[8em][c]{Tomasz Pieczerak}
}

\addtolength{\oddsidemargin}{-.3in}
\addtolength{\evensidemargin}{-.3in}
\addtolength{\textwidth}{0.6in}
\addtolength{\topmargin}{-.75in}
\addtolength{\textheight}{1.5in}

\begin{document}
\maketitle

\section{Treść zadania}
Zaimplementować sieć typu master-slave wykorzystując jako warstwę fizyczną
łącze RS485. Powinny być dostępne następujące usługi warstwy łącza danych:

\begin{itemize}
  \item Send Data with Acknowledge (SDA);
  \item Send Data with No Acknowledge (SDN) -- usługa typu broadcast.
\end{itemize}

\paragraph{Założenia}\

\begin{itemize}
  \item każdy węzeł sieci posiada unikalny identyfikator (liczba 8-bitowa),
  \item maksymalna liczba węzłów w sieci wynosi $64$,
  \item w sieci znajduje się jeden węzeł (\emph{master}) zarządzający
        dostępem do łącza oraz wiele węzłów podrzędnych (\emph{slave}),
  \item węzeł zarządzający (\emph{master}) zna adresy wszystkich węzłów
        podłączonych do sieci (ale niektóre z nich mogą być wyłączone).
\end{itemize}

Szczególną uwagę należy zwrócić na obsługę błędów (należy wprowadzić
mechanizmy symulowania błędów). Implementacja ma zostać wykonana w środowisku
GNU/Linux.

\section{Dodatkowe założenia}

W trakcie analizy treści zadania dostrzegliśmy potrzebę wprowadzenia pewnych
dodatkowych założeń/ograniczeń precyzujących implementowane zagadnienie:

\begin{itemize}
  \item rozwiązanie \textbf{nie musi} być optymalne -- naszym celem nie jest
        stworzenie najbardziej wydajnej sieci, lecz sieci \emph{po prostu
        działającej};
  \item nie implementujemy procesu rejestracji maszyn, tj. każdy host typu
        \emph{slave} posiada przypisany na stałe adres, o którym wiedzę
        posiada \emph{master};
  \item dostarczone funkcje obsługujące wysyłanie/odbieranie danych na/z
        szyny opierają się na komunikacji znakowej, toteż nie ma sensu żmudne
        budowanie nagłówków ramek z pojedynczych bitów; operować będziemy
        \emph{zawsze} na bajtach;
  \item jako że pracujemy w warstwie sieciowej nie musimy dbać o poprawną
        interpretację danych zawartych w pakietach, naszym zadaniem jest
        tylko i wyłącznie zapewnienie poprawnego przesyłu pakietów, składanie
        tychże w sensowne komunikaty leży w gestii oprogramowania warstw
        wyższych;
  \item konieczna jest weryfikacja poprawności przesyłanych danych,
        w szczególności należy dodać do nagłówka sumę kontrolną pakietu
        (\emph{CRC});
  \item \emph{master} jest wyłącznie węzłem kontrolującym szynę -- nie
        odbiera ani nie wysyła żadnych pakietów danych.
\end{itemize}

\section{Logika protokołu}

\paragraph{Przydział łącza}\

Przydział łącza przez maszynę \emph{master} opiera się na algorytmie
\emph{Round Robin}. Węzeł zarządzający odpytuje kolejno wszystkie maszyny
\emph{slave}, wysyłając do nich ramki \texttt{BUS}. Domyślnie \emph{master}
zakłada, iż odpytywany węzeł będzie transmitował dane. Jeśli \emph{slave} nie
ma danych do przesłania, może on przesłać do węzła zarządzającego ramkę
\texttt{NRQ} w celu rezygnacji z prawa do transmisji. Analogiczny skutek będzie
miał brak odpowiedzi - po przekroczeniu ustalonego timeoutu \emph{master}
przekaże prawo do transmisji następnej maszynie. W ramach pojedynczej
transmisji można wysłać do 10 bajtów danych. Następnie \emph{master} ponownie
przejmuje szynę i kontynuuje odpytywanie od następnej maszyny.

\paragraph{Przesył danych (z punktu widzenia użytkownika)}\

Maszyna \emph{slave} chcąc przesłać dane nasłuchuje na szynie czekając na
swoją kolejkę -- tj. czeka na ramkę \texttt{BUS} adresowaną do niej. Po
otrzymaniu takowej rozpoczyna transmisję, tzn. wysyła \textbf{jedną} ramkę
\texttt{DAT} o zmiennej długości pola danych od 1 do 10 bajtów. W przypadku
transmisji z potwierdzeniem po wysłaniu danych stacja oczekuje na
potwierdzenie, a następnie zwalnia szynę. Jeśli zostały jeszcze dane do
wysłania stacja czeka na ponowny przydział łącza i po jego otrzymaniu powtarza
procedurę.

Dodatkowo ramki \texttt{DAT} w trybie SDA są numerowane. W przypadku braku
otrzymania potwierdzenia poprawnego odbioru danych transmisja zostanie
przerwana, zaś aplikacja otrzyma informację o ilości danych, które zostały
poprawnie przesłane. W takim przypadku pierwsza ramka kolejnej transmisji
danych będzie miała numer ostatniej niepotwierdzonej ramki \texttt{DAT}.

Retransmisja zagubionych danych nie leży w gestii protokołu, jednak aby
zapobiec duplikowaniu ramek \texttt{DAT} u odbiorcy w przypadku zagubienia
potwierdzenia kiedy to aplikacja retransmituje dane, protokół zapewnia, iż
otrzymanie kolejno dwóch ramek danych o takim samym numerze spowoduje
zignorowanie danych z drugiej ramki (chociaż zostanie ona potwierdzona).
Dodatkowo każdy węzeł utrzymuje indywidualne liczniki ramek dla każdego z
pozostałych węzłów (tzn. każda para maszyn pracujących w sieci posiada
indywidualną parę liczników), co zwalnia maszyny z obowiązku śledzenia
transmisji, które nie są adresowane do nich.

\paragraph{Kwestia \emph{timeout'ów}}\

Istnieją dwie sytuacje, w których konieczne jest wprowadzenie czasowego
przeterminowania odpowiedzi na ramki, aby zapobiec zawieszeniu działania
protokołu z powodu nieobecności którejś ze stacji:
\begin{itemize}
  \item odpowiedĽ \emph{slave'a} na ramkę \texttt{BUS},
  \item potwierdzenie \texttt{ACK} odebrania danych przez \emph{slave'a}
        (odbiorcę).
\end{itemize}
Aby uprościć tworzenie protokołu zamiast mierzyć rzeczywisty upływ czasu
uznaliśmy, że wystarczającym kryterium będzie pojawienie się bajtu
oznaczającego początek ramki w pewnej ustalonej liczbie odebranych bajtów po
rozpoczęciu oczekiwania. W obu wymienionych powyżej przypadkach uznajemy, że
jeśli w pierwszych \textbf{5 bajtach} nie pojawi się znacznik początku
poprawnej ramki, nasłuchujący może uznać, iż odpowiedzi nie było.

\paragraph{Weryfikacja poprawności datagramów}\

Każda wysyłana ramka jest opatrzona \emph{cyklicznym kodem nadmiarowym}
\emph{CRC16}. Kod funkcji obliczającej sumę kontrolną został dostarczony
przez prowadzących. Ponadto 

\section{Formaty ramek}

Ogólny format ramek wykorzystywanych przez protokół wygląda następująco:
\begin{displaymath}
  \underbrace{\hbox{<<znacznik początku ramki>>}}_{1B}
  \underbrace{\hbox{<<CRC16>>}}_{2B}
  \underbrace{\hbox{<<etykieta typu>>}}_{1B}
  \underbrace{\hbox{<<zależne od typu>>}}_{0-14B}
\end{displaymath}

\emph{Znacznik początku ramki} o wartości numerycznej $0xBF$ służy łatwiejszej
identyfikacji ramek w strumieniu bajtów i jest wspólny dla wszystkich typów
(wprowadzamy dla niego oznaczenie \texttt{BOF} -- \emph{begin of frame}).
W protokole występują cztery typy ramek (w nawiasach podane jest jaki rodzaj
stacji może daną ramkę wysłać, a jaki odebrać):
\begin{itemize}
  \item BUS (\emph{master} $\rightarrow$ \emph{slave}) -- zapytanie
        o zainteresowanie szyną
    \begin{displaymath}
      \underbrace{\hbox{<<BOF>>}}_{1B}
      \underbrace{\hbox{<<CRC16>>}}_{2B}
      \underbrace{01h}_{1B}
      \underbrace{\hbox{<<adres \emph{slave'a}>>}}_{1B}
    \end{displaymath}
  \item NRQ (\emph{slave} $\rightarrow$ \emph{master}) -- rezygnacja z dostępu do
        szyny (wysyłanie pakietów \texttt(NRQ) przez odpytywaną maszynę pracującą
		w trybie \emph(slave) jest opcjonalne).
    \begin{displaymath}
      \underbrace{\hbox{<<BOF>>}}_{1B}
      \underbrace{\hbox{<<CRC16>>}}_{2B}
      \underbrace{02h}_{1B}
    \end{displaymath}
    Adres stacji rezygnującej z dostępu nie jest tu konieczny, gdyż \emph{master}
	wie komu przyznał dostęp.
  \item DAT (\emph{slave} $\rightarrow$ \emph{slave}) -- pakiet danych
    \begin{displaymath}
      \underbrace{\hbox{<<BOF>>}}_{1B}
      \underbrace{\hbox{<<CRC16>>}}_{2B}
      \underbrace{04h}_{1B}
      \underbrace{\hbox{<<numer ramki>>}}_{1B}
      \underbrace{\hbox{<<adres odbiorcy>>}}_{1B}
      \underbrace{\hbox{<<adres nadawcy>>}}_{1B}
      \underbrace{\hbox{<<długość>>}}_{1B}
      \underbrace{\hbox{<<dane>>}}_{1-10B}
    \end{displaymath}
    Pole \emph{długość} informuje od rozmiarze pola danych. Adres odbiorcy
    $0xFF$ oznacza transmisję \emph{broadcast'ową} bez potwierdzenia.
  \item ACK (\emph{slave} $\rightarrow$ \emph{slave}) -- potwierdzenie
        otrzymania danych
    \begin{displaymath}
      \underbrace{\hbox{<<BOF>>}}_{1B}
      \underbrace{\hbox{<<CRC16>>}}_{2B}
      \underbrace{08h}_{1B}
      \underbrace{\hbox{<<numer potwierdzanej ramki>>}}_{1B}
    \end{displaymath}
    Adres nie jest konieczny, nadawca wie do kogo wysyłał i nie są możliwe
    współbieżne transmisje, więc ramka zawsze odnosi się do aktualnie
    trwającej transmisji.
\end{itemize}

\section{Architektura rozwiązania}

\paragraph{Funkcje protokołu}\

Protokół komunikacji będzie udostępniał kilka funkcji pozwalających na jego
użytkowanie z poziomu aplikacji.
\begin{itemize}
  \item \texttt{mss\_run\_master(\textnormal{<<struktura sieci>>})} : funkcja
        wywoływana tylko i wyłącznie w węĽle \emph{master} i realizująca
        zaprojektowany algorytm przydziału łącza, \emph{master} zna adresy
        wszystkich stacji w sieci, stąd argument je opisujący;
  \item \texttt{mss\_slave\_send(\textnormal{<<adres>>}, \textnormal{<<dane>>})}
        : wywoływana w węĽle \emph{slave} celem transmisji danych do innej
        (lub innych) stacji, funkcja jest blokująca -- do momentu zakończenia
        transmisji, wartość zwracana informuje o powodzeniu operacji;
  \item \texttt{mss\_slave\_recv(\textnormal{<<bufor>>})} : po wywołaniu tej
        funkcji węzeł \emph{slave} zaczyna nasłuchiwać na szynie transmisji
        adresowanej do niego, odbiera dane do wskazanego bufora odsyłając
        w razie potrzeby potwierdzenie \texttt{ACK} do nadawcy.
\end{itemize}
Funkcja wysyłająca zawiera w sobie całą logikę działania protokołu --
oczekiwanie na ramkę \texttt{BUS}, odpowiedĽ ramką \texttt{REQ} i transmisję.
W przypadku błędu lub braku potwierdzenia nie próbuje ponownie wysłać danych
-- leży to w gestii aplikacji.

\paragraph{Aplikacje}\

Przygotować należy dwa programy w języku C: program dla węzła \emph{master}
oraz dla węzła \emph{slave}:
\begin{itemize}
    \item \texttt{mss-bus-master} to program działający na węĽle
          \emph{master}, utrzymuje on statyczną tablicę adresów hostów
          \emph{slave} i kontroluje dostęp do szyny, kod programu sprowadza
          się w zasadzie do wywołania funkcji \texttt{mss\_run\_master()};
    \item \texttt{mss-bus-slave} to program użytkownika, który będzie działał
          dwóch trybach (zależnie od podanych parametrów):
          \begin{itemize}
              \item odbieranie : stacja będzie oczekiwała na dane adresowane
              do niej, odbierała je i wypisywała na ekran (użytkowana tu
              będzie funkcja \texttt{mss\_slave\_send()}),
              \item wysyłanie : program pobierze dane do wysłania (z
              \texttt{stdin} lub pliku) i za pomocą funkcji
              \texttt{mss\_slave\_send()} wyśle je do wskazanej stacji lub
              w trybie \emph{broadcast}.
          \end{itemize}
\end{itemize}

\section{Projekt testowania}

Testy odbędą się w sali 138, bo tam znajduje się potrzebny sprzęt.
Przetestowane zostaną przede wszystkim sytuacje awaryjne, w szczególności
odłączanie węzłów w trakcie transmisji/poza transmisją, wyłączanie nadawców
i odbiorców, wyłączanie i włączanie urządzeń, zakłócenia.

Poza tym podczas testów będziemy w miarę potrzeb dostosowywać parametry
protokołu m.in. \emph{timeout'y}.

\end{document}

